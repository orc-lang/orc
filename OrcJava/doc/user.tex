\documentclass{article}
 \topmargin -1.5cm        % read Lamport p.163
 \oddsidemargin -0.04cm   % read Lamport p.163
 \evensidemargin -0.04cm  % same as oddsidemargin but for left-hand pages
 \textwidth 16.59cm
 \textheight 21.94cm 
% \parskip 7.2pt           % sets spacing between paragraphs
%\renewcommand{\baselinestretch}{1.5} 	% Uncomment for 1.5 spacing between lines
% \parindent 0pt		  % sets leading space for paragraphs

\usepackage{listings}

\usepackage{latexsym}
\usepackage{amsmath}
\usepackage{alltt}
\newenvironment{alltts}
   {\begin{alltt}} %\footnotesize
   {\end{alltt}}

\newcommand{\s}[1]{{\it{#1}}}
\newcommand{\B}[1]{``{\tt{#1}}''}
\newcommand{\Q}[1]{``{\tt{#1}}''}


\lstdefinelanguage{orc}{%
  numbers=none,
  morekeywords={def,::=},
  sensitive=true,
  morecomment=[l]{//},
  morecomment=[s]{/*}{*/},
  morestring=[b]",
}

\lstset{%
  frame=none,
  xleftmargin=5pt,
  stepnumber=1,
  numbers=left,
  numbersep=5pt,
  numberstyle=\ttfamily\tiny\color[gray]{0.3},
  belowcaptionskip=\bigskipamount,
  captionpos=b,
  escapeinside={*'}{'*},
  language=orc,
  tabsize=2,
  emphstyle={\bf},
  commentstyle=\it,
  stringstyle=\mdseries\rmfamily,
  showspaces=false,
  identifierstyle=\it,
  keywordstyle=\tt,
  columns=flexible,
  basicstyle=,
  showstringspaces=false,
  morecomment=[l]\%,
}


\begin{document}
\bibliographystyle{plain}

\title{Orc 0.5 Users Guide\footnote{Copyright 2005, The University of Texas at Austin. All rights reserved.
}}

\author{William R. Cook and Javadev Misra}

\maketitle

%\begin{abstract}
%\end{abstract}

%\begin{quotation}
% \end{quotation}

%\newpage
%\tableofcontents 

\section{Introduction}
\label{introduction}

This document describes the experimental Orc programming system.
It describes how to install the system and execute programs. 
It also defines the current concrete syntax of Orc
that is recognized by the system. This document assumes
a working knowledge of the purpose and semantics of Orc;
these are described in other documents
\cite{Orc-Marktoberdorf04,Cook-Misra-Orc,Cook-Misra-OrcSem}.

The Orc system is under active development; it currently does not
have nearly all the features that we would like. However, the
current implementation is a very clean foundation on which to build
these additional features.

The source code is publicly available under GPL. The system
is written in Java and includes javadoc documentation. In 
addition, a high-level description of the implementation
is also available \cite{Cook-Misra-OrcImp}.

\section{Installation}
\label{Installation}

Unpack the {\tt orc.zip} file into any directory on your system. 
The resulting directory structure is as follows:

\begin{tabular}{ll}
{\tt orc			} & command to run Orc on unix \\
{\tt orc.bat		} & command to run Orc on windows \\
{\tt README.txt		} & readme file \\
{\tt license.txt	} & license agreement \\
{\tt doc/			} & documentation (including this document) \\
{\tt lib/			} & executable libraries (jar files) used by Orc \\
{\tt test/		    } & example scripts
\end{tabular}

\section{Running Orc}

To run Orc, simply execute the command

{\tt orc} \s{filename}

\noindent
where \s{filename} is a file containing an Orc program.
The file contains a set of definitions  and one goal expression.

To enable Orc to be run from any directory, use one of the following:
\begin{itemize}
\item Add the directory containing the {\tt orc} command to your path.
\item Create an alias to the appropriate file ({\tt orc} on Unix or {\tt orc.bat} on Windows).
\item Specify the path of the Orc command, as in {\tt myorc/orc}.
\end{itemize}

Orc reads from the standard input if no filename is given. This can be used to execute
Orc from within Emacs.

{\tt orc} {\tt <} \s{filename}

Debugging information can be printed using the {\tt -debug} flag as the first argument.

\section{Examples}

\subsection{Metronome}
\begin{verbatim}
def Metronome(x) = let(true) | Rtimer(x) >>  Metronome(x)

Metronome(1000)
\end{verbatim}

\subsection{Random Delay}
\begin{verbatim}
-- Rdelay returns a random number x between 0 and 5000 after x millsecs.
-- change 5000 below to get diffrent delays.

def Rdelay =  random(5000) >x> Rtimer(x) >> let(x)
\end{verbatim}

\subsection{Parallel Or}
\begin{verbatim}
def parallelOr(M, N) =
	let(z)
	where z in { if(x) | if(y) | or(x, y) };
		  x in M;
		  y in N

-- some tests, using random delays
def A = Rtimer(3000) >> let(true)
def B = Rtimer(1000) >> let(false)
def C = Rtimer(500) >> let(true)

parallelOr(A, B) 
>!> 
parallelOr(A, C)
>!> 
parallelOr(B, B)
\end{verbatim}


\section{Sites}

The following sites can be called:

\begin{description}
\item[Tuples]\  
\begin{description}

\item[Let:]
{\tt let($e_1$, $e_2$, $...$, $e_n$)}

Produces a tuple of values. Let is strict, in that it waits for
all the arguments to be defined.

\item[Item:]
{\tt item($x$, $n$), item($s$, $n$, $m$)}

Access the $n$th item of a tuple or string. Indexes are zero-based.
For strings, item can take an extra argument to select a substring 
of characters from $n$ upto $m$. Thus 

{\tt let(1, 2) >x> item(x, 0)} 

produces {\tt 1} and 

{\tt item("foo", 1, 2)} 

produces {\tt "o"}.

\end{description}

\item[Arithmetic]\  
\begin{description}
\item[Binary Operators:]
	{\tt add}, 
	{\tt sub}, 
	{\tt mul}, 
	{\tt div}

For example, {\tt add(3, sub(9, 2))}.

\item[Binary Comparisons:]
{\tt lt}, 
{\tt le}, 
{\tt eq}, 
{\tt ne}, 
{\tt ge}, 
{\tt gt} 

For example, {\tt lt(3, 9)} is true.

\end{description}

\item[Logical]\ 
\begin{description}
\item[Constants:]
{\tt true},
{\tt false}

\item[Unary:]
{\tt not}

\item[Binary:]
{\tt and},
{\tt or}

\item[Control:]
{\tt if($b$)}

If outputs a value when $b$ is true. It does not produce a value
when $b$ is false. 
\end{description}

\item[Output:]\ 
{\tt print($e_1$, $e_2$, $...$, $e_n$)},
{\tt println($e_1$, $e_2$, $...$, $e_n$)}

Prints arguments to standard output.

\item[Mail:]
{\tt SendMail({\it from}, {\it to}, {\it subject}, {\it content}, {\it server})}

Sends a mail message via an SMTP server. Does not currently support servers 
that require login. The {\it to} field can be a list created by {\tt let}.
For example:
\begin{verbatim}
SendMail("wcook@cs.utexas.edu", 
         let("misra@cs.utexas.edu", "wcook@cs.utexas.edu"),
         "this is a test", 
         "I think it is working now", 
         "mail.cs.utexas.edu")
\end{verbatim}
		 
\item[Time:]
{\tt Rtimer($n$)}

Wait $n$ milliseconds, the return 1.

\item[Miscellaneous:]
{\tt random($n$)}

Returns a random number between $0$ and $n$.

\end{description}

\subsection{Nested Calls}
Site calls can be nested. Thus the following is legal:

{\tt if(lt(x, 10)) >> ... }

The meaning of a nested site call is:

{\tt M(A(), B())}

is interpreted as 

{\tt M(a, b) where a in A(); b in B()}

Note: currently all calls can be nested, not just pure functional calls.

\section{Grammar}

The syntax of Orc is defined by the extended BNF grammar in Figure~\ref{syntax}.
Keywords and symbols that appear in the program are quoted. There
are three keywords: \B{def}, \B{where} and \B{in}. The special 
symbols are \Q{(}, \Q{)}, \Q{=}, \Q{,}, \Q{;}, \Q{>}, \Q{!}, \Q{|}, \Q{\{} and \Q{\}}.

Names in italics are nonterminals. The form 
\{ \s{x} \} denotes zero or more occurences of \s{x}, while
[ \s{x} ] denotes zero or one occurences of \s{x}.

There are several difference from the syntax of Orc in theoretical papers:

\begin{itemize}
\item Definitions can be given at any level, and can be nested following block structure.
\item Calls can be nested inside other calls.
\item There is one namespace for variables and site names. Sites can be passed
and returned as values (although this is not been tested).
\end{itemize}

NOTE: These differences may be corrected as the implementation is improved.

\begin{figure}
\begin{tabular}{lcl}
\s{expr} &::=& \{ \s{def} \} \s{goal} 
\\\\
\s{def} &::=& \B{def} \s{id} [ \s{formals} ] \Q{=} \s{expr}
\\[3pt]
\s{formals} &::=& \Q{(} \s{id} \{ \Q{,} \s{id} \} \Q{)}
\\\\
\s{goal} &::=& \s{par} [ \B{where} \s{binding} \{ \Q{;} \s{binding} ]
\\[3pt]
\s{binding} &::=& \s{id} \B{in} \s{par}
\\\\
\s{par} &::=& \s{seq} \{ \Q{|} \s{seq} \}
\\\\
\s{seq} &::=& \s{basic} \{ \Q{>} [\Q{!}] [\s{id}] \Q{>} \s{basic}  \}
\\\\
\s{basic} &::=& \s{id} $\mid$ \s{number} $\mid$ \s{string} $\mid$ \s{call} $\mid$ \s{block}
\\[3pt]
\s{id} &::=& [a-zA-Z] \{ [a-zA-Z0-9] \}
\\[3pt]
\s{number} &::=& \{ [0-9] \}
\\[3pt]
\s{string} &::=& \s{Java-style string literal}
\\\\
\s{call} &::=& \s{id} \Q{(} \s{expr} \{ \Q{,} \s{expr} \} \Q{)}
\\\\
\s{block} &::=& \Q{\{} \s{expr} \Q{\}} 
\\\\
\s{comment} &::=& \Q{--} \s{any text}
\end{tabular}
\caption{\label{syntax}Syntax of Orc 0.5}
\end{figure}

%\end{alltts}

\section{Planned Enhancements}

Some planned enhancements, in rough priority order from most important to least.

\begin{itemize}

\item Allow "import" of a file into another file, i.e., textual inclusion.

\item Allow commenting large regions using, say, {- -}

\item Allow infix arithmetic and logical expressions as arguments of site
  calls. These should only allow side-effect free,
  deterministic, immediate site calls. An easy solution is to convert
  the expression to prefix form since you already support it.

\item Receive email responses.

\item Interface to web services:
	Plan to use Apache Axis2 to support asynchronous calls.

\item Acces to java objects. Add se of dot notation for records/objects, so that 
  {\tt x.m(args)} will call the {\tt m} method of {\tt x}.
  
\item Ability to define local sites and use them. This and a few other things
	can be defined as short-hand forms of existing Orc behavior.

\item Misc sites:
	Clock, 
	Cell, 
	Word,
	Persistent storage, 
	Access to files,
	Input, Output.
	Cell and Word are blocking and non-blocking storage cells.
	

\end{itemize}

\bibliography{references}

\end{document}
